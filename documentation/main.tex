\documentclass[parskip=half]{scrartcl}

% Language
\usepackage{polyglossia}
\setdefaultlanguage{english}

% Font
%\usepackage{textcomp}
%\usepackage{fontspec}
%\usepackage{unicode-math}
%\setmainfont{Linux Libertine O}
%\setmathfont{Libertinus Math}
%\setsansfont{Linux Biolinum}
%\usepackage{microtype}

\usepackage{tikz}
\usepackage{tcolorbox}
\usepackage{listings}
\lstset{%
	basicstyle=\ttfamily%,
	%keywordstyle=\bf\color{blue},
	%commentstyle=\color{red},
	%stringstyle=\color{green!60!black} ,
	%identifierstyle=,
	%language=c++,
	%showstringspaces=false,
	%breaklines=true,
	%breakatwhitespace=true,
	%morecomment=[s]{\%\{}{\%\}},
	%morecomment=[s]{\#\{}{\#\}},
	%morekeywords={sc_module},
    %showlines=false,
    %xleftmargin=0.7cm,
    %resetmargins=false
}

\input{resistor}
\input{capacitor}

\title{Circuits Package}
\subtitle{Roadmap}
\author{Patrick Schulz}
\date{\today} % replace with release date

\begin{document}
    \maketitle
    \tableofcontents
    \section{Overview}
    \begin{lstlisting}
    \begin{circuit}
        [
            template = divider
        ]

    \end{circuit}
    \end{lstlisting}
    \section{Symbols}
    \subsection{The \texttt{\textbackslash component} command}
    The \lstinline!\component! command is used to place components (single instances or compounds of devices) for further use especially in circuit templates 
    (see section \ref{sec:templates}).
    \begin{tcolorbox}
        \lstinline!\component{R}!
        \tcblower
        \centering
        \begin{tikzpicture}
            \node[resistor] { };
        \end{tikzpicture}
    \end{tcolorbox}
    Multiple devices can be used:
    \begin{tcolorbox}
        \lstinline!\component{R\\C}!
        \tcblower
        \centering
        \begin{tikzpicture}
            \node[resistor] { };
            \node[capacitor] at (0, -0.75) { };
        \end{tikzpicture}
    \end{tcolorbox}
    \begin{tcolorbox}
        \lstinline!\component[draw wires]{R\\C,R}!
        \tcblower
        \centering
        \begin{tikzpicture}
            \node[resistor] (R1) { };
            \node[capacitor] at (-0.5, -0.75) (C1) { };
            \node[resistor] at ( 0.5, -0.75) (R2) { };
            \draw[wire] (C1.east) -- (R2.west);
            \draw[wire] (C1.west) -- ++(-0.25, 0) |- (R1.west);
            \draw[wire] (R2.east) -- ++( 0.25, 0) |- (R1.east);
        \end{tikzpicture}
    \end{tcolorbox}
    \lstinline!\\! binds weaker than \lstinline!,!, so in the following examples the devices have to be grouped:
    \begin{tcolorbox}
        \lstinline!\component[draw wires]{{R\\C},R}!
        \tcblower
        \centering
        \begin{tikzpicture}[dot/.style = {fill, circle, inner sep = 0cm, minimum size = 0.1cm}]
            \node[resistor] at(0, 0.375) (R1) { };
            \node[capacitor] at (0, -0.375) (C1) { };
            \node[resistor] at (1, 0) (R2) { };
            \draw[wire] (C1.west) -- ++(-0.4, 0) |- (R1.west);
            \draw[wire] (C1.east) -- ++( 0.4, 0) |- (R1.east);
            \draw[wire] ([xshift=0.4cm] C1.east |- R2.west) node[dot] { } -- (R2.west);
        \end{tikzpicture}
    \end{tcolorbox}
    \section{Circuit Templates}\label{sec:templates}
    \subsection{Transistor Circuits}
    \subsection{Operational Amplifier Circuits}
    \subsection{Passive Circuits}
    \section{Module System}
\end{document}

% vim: ft=tex
